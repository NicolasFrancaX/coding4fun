\documentclass[a4paper,11pt,fleqn]{article}

\usepackage[brazil]{babel}
\usepackage[utf8]{inputenc}
\usepackage{xspace}
\usepackage{color}
\usepackage{graphicx}
\usepackage{hyperref}
\usepackage{enumitem}

\usepackage{calligra}
\DeclareMathAlphabet{\mathcalligra}{T1}{calligra}{m}{n}


\newcommand{\blue}[1]{\textcolor{blue}{#1}}

\title{Relatório - Placar 1\\
Desafios de Programaçao - MC521}
\author{Aluno: Nicolas França\\\\
Professor: Fábio Luiz Usberti\\\\
Unicamp - Instituto de Computação\\}
\date{}

\begin{document}

\maketitle

\section{C}
\label{s:c}

Nos é dado o $\mathcallibra{a}$, $\mathcallibra{b}$ e $\mathcallibra{c}$ de uma 
  \textbf{Equação Diofantina Linear}. Sabemos que é uma \textbf{Equação Diofantina Linear} pois $\mathcallibra{a}$, $\mathcallibra{b}$ e 
  $\mathcallibra{c}$ são números inteiros e $\mathcallibra{x}$ e $\mathcallibra{y}$ só
  podem ser inteiros.

Sabemos que uma \textbf{Equação Diofantina Linear} só pode ser resolvida \textbf{se}
  $\mathcallibra{c}$ é divisível por $\mathcallibra{gcd(a, b)}$. Logo, se calcularmos
  o $\mathcallibra{gcd(a, b)}$ e observarmos que o $\mathcallibra{c}$ é divisível por ele,
  então podemos imprimir \textbf{Yes} na tela. Do contrário, imprimimos \textbf{No}.


\section{H}
\label{s:h}

Premissa: quase todos os números são divisíveis por 2.

Então se tivermos um intervalo com 1, 2, 3, vemos que o número 2 é o divisor que aparece mais
  vezes na sequência. Agora se tivermos o valor 3 como $\mathcallibra{l}$ e como $\mathcallibra{r}$,
  então o número 3 é o divisor que mais aparece na sequência.
  No caso em que o valor 9 é assumido tanto em $\mathcallibra{l}$ e em $\mathcallibra{r}$, como podemos
  usar valores quaisquer desde que seja o divisor que mais aparece, podemos simplesmente mostrar o 
  valor 9 mesmo.

Ou seja. \textbf{Se} $\mathcallibra{l}$ = $\mathcallibra{r}$, apresentamos o valor $\mathcallibra{l}$ na tela. \textbf{Se não}, apresentamos
  o valor 2.



\bibliographystyle{plain}
\bibliography{refs}

\end{document}
